\section{User-Defined Types}
\label{sec:userdefined}

\subsection{Type synonyms}
\label{sec:synonyms}

\begin{wrapfigure}{r}{0.32\columnwidth}
\vspace{-2em}
\begin{mscode}
typedef Bit#(8) Byte;
Byte x = 8'd1;
\end{mscode}
\vspace{-2em}
\end{wrapfigure}

Type synonyms allow giving a different name to an existing type. Their syntax is:
\begin{center}
\verb|typedef Type NewType;|
\end{center}
where \verb|Type| is any existing type, and \verb|NewType| is the new type's name.
Type synonyms are a convenience feature. The original type and its synonym can be used interchangeably,
e.g., without any type conversions on assignments.

Type synonyms can be parametric; see \autoref{sec:parametrics} for details.

\subsection{Structs}
\label{sec:structs}

\begin{wrapfigure}{r}{0.32\columnwidth}
\vspace{-2em}
\begin{mscode}
// Definition
typedef struct {
  Byte red;
  Byte green;
  Byte blue;
} Pixel;

// Creation
Pixel cyan = Pixel{
  red : 0,
  green : 255,
  blue : 255
};

// Member access
Bit#(10) intensity =
  zeroExtend(cyan.red) +
  zeroExtend(cyan.green) +
  zeroExtend(cyan.blue);

// Assignment
Pixel white = cyan;
white.red = 255;
\end{mscode}
\vspace{-6em}
\end{wrapfigure}

Structs are composite types (also known as product types):
they represent a group of \emph{members} of different types.

\paragraph{Definition:} Struct definitions use the following syntax:
\begin{center}
\verb|typedef struct {| \\
\verb|  Type1 member1;| \\
\verb|  Type2 member2;| \\
\verb|      ...       | \\
\verb|  TypeN memberN;| \\
\verb|} StructType;   |
\end{center}
where \verb|StructType| is the (new) type for the struct,
\verb|member|$_{\texttt{i}}$ are the (lowercase) names of its members, and
\verb|Type|$_{\texttt{i}}$ are the (uppercase) types of its members.

Structs can be parametric; see \autoref{sec:parametrics} for details.

\paragraph{Creation:} The struct creation expression (\autoref{sec:structExpr}) allows constructing a struct value.
Its syntax is:
\begin{center}
  \verb|StructType { member1 : expr1, ..., memberN : exprN }|
\end{center}
where \verb|StructType| is the struct's type name, \verb|member|$_{\verb|i|}$ are the struct members' names,
and \verb|expr|$_{\verb|i|}$ denote the values that the members should take.

\paragraph{Member access:} Given value \verb|s| of a struct type that has a member with name \verb|m|,
the expression \verb|s.m| yields the value of member \verb|m|.
If \verb|s| is a variable, then values can be assigned to its individual members as shown in the example.

\begin{wrapfigure}{r}{0.32\columnwidth}
\vspace{-3.5em}
\begin{mscode}
// Enum definition
typedef enum {
  Ready, Busy, Error
} State;

// Usage example; assume
// that mod is a module
State s = mod.getState();
if (state == Ready)
  // Feed mod an input
else if (state == Error)
  // Halt system

// Enum def with label values
typedef enum {
 Red = 0, Blue = 2, Green = 1
} PixelChannel;
\end{mscode}
\vspace{-5em}
\end{wrapfigure}

\subsection{Enums}
\label{sec:enums}

Enums or enumerations represent a set of unique symbolic constants, called \emph{labels}.
Enums can be defined using the following basic syntax:
\begin{center}
  \verb|typedef enum { Label1, ..., LabelN } EnumType;|
\end{center}
where \verb|EnumType| is the enum's type name, and \verb|Label|$_{\verb|i|}$ are the names of the labels.
Labels must be uppercase, and can be repeated across enum definitions.
A value of type \verb|EnumType| can take one of these labels.

The compiler internally represents an enum with $N$ possible labels as a $\lceil log_2N \rceil$-bit value.
With the syntax above, \verb|Label|$_{\verb|i|}$ will take numeric value \verb|i-1| (i.e., labels take consecutive values starting from \verb|0|).
It is possible to assign the numeric value of each label explicitly using the following syntax:
\begin{center}
  \verb|typedef enum { Label1 = val1, ..., LabelN = valN } EnumType;|
\end{center}
where \verb|val|$_{\verb|i|}$ are the distinct numeric values of the labels.
